% unicodeは、hyperrefへの指定で、pdfのメタデータにあるタイトルの文字化けを防ぐ
\documentclass[unicode, 14pt]{beamer} 
\usepackage{listings}
\usepackage{bussproofs}
\usepackage[backend=biber, style=ieee]{biblatex}
\usetheme{rikako}
\addbibresource{demo.bib}
\title{はじめてのrunc}
\author{中村 遼太郎}
\begin{document}

\begin{frame}[noframenumbering, plain]
\titlepage
\end{frame}
\section{導入}
\begin{frame}[t]
  \frametitle{runcってなに}
  低レベルなコンテナランタイムとそのCLI
  % 左にレイヤの図。右に箇条書
\end{frame}
\begin{frame}[t]
  \frametitle{コンテナを起動}
  コマンドを実行すれば低レベルさがみえる
  % コンテナを起動するコード
\end{frame}
\begin{frame}[t]
  \frametitle{runcの実装}
  runcは既存の小さなコンポーネントを再利用
  % Principle
\end{frame}
\begin{frame}[t]
  \frametitle{runc runの手続き}
  runが生成したinitがコンテナになる
\end{frame}
\section{runc run}
\begin{frame}[t]
  \end{frame}
\section{runc init}

\begin{frame}[t]
  \frametitle{a}
\end{frame}
% runcの寄贈
% https://www.docker.com/blog/runc/
% https://www.docker.com/blog/containerd-vs-docker/
% https://docs.docker.com/engine/daemon/alternative-runtimes/
\section{特徴}
\begin{frame}[t]
  \frametitle{特徴}
    単純なスライドにした。\textit{abc}
  \vspace{0.2\paperheight}
  \begin{itemize}
    \item 右下のnavigationを無効化\href{https://google.com}{doge}
    \item 各セクションの先頭に自動で目次を挿入
    \item frameが複数のスライドにまたいでも\\タイトルの末尾に番号をつけない
  \end{itemize}
\end{frame}

% \section{フォント}

\begin{frame}
\frametitle{採用したフォント}
\begin{itemize}
\item ヒラギノ
\item Helvetica Neue
\item Source Han Code JP
\item JetBrains Mono
\item STIX2
\end{itemize}
\end{frame}

\section{デモ}

\begin{frame}
  \frametitle{数式の例}
  \begin{equation}
    e^{i\theta} = \cos\theta + i\sin \theta \mathrm{sin}
  \end{equation}
\end{frame}

\begin{frame}
  \frametitle{導出木の例}
  \begin{prooftree}
    \AxiomC{$P \to Q$}
    \AxiomC{$P$}
    \BinaryInfC{$Q$}
  \end{prooftree}
\end{frame}

\begin{frame}[fragile]
\frametitle{コード}
  {\small 
    \begin{lstlisting}[language=C++]
#include<bits/stdc++.h>
using namespace std;

int main() {
  cout << "Hello World!" << endl;
  return 0;
}
    \end{lstlisting}
}
\end{frame}

\begin{frame}[allowframebreaks]
  \frametitle{参考文献の表示例}
  \printbibliography
  % 引用していないbibファイルの要素も記載する  
  \nocite{*} 
\end{frame}

\end{document}
