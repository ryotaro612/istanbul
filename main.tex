% unicodeは、hyperrefへの指定で、pdfのメタデータにあるタイトルの文字化けを防ぐ
% ptは細かい指定はできないらしい
% チートシート
% https://www.cpt.univ-mrs.fr/~masson/latex/Beamer-appearance-cheat-sheet.pdf
\documentclass[unicode, 14pt, aspectratio=169]{beamer} 
\usepackage{minted}
\usepackage{listings}
\usepackage{enumitem}
\usepackage{xcolor}
\usepackage{textcomp}
\usepackage[backend=biber, style=ieee]{biblatex}
\usetheme{rikako}
\date{\number\year 年\number\month 月\number\day 日}

\addbibresource{main.bib}
\title{runcのUNIXプログラミング}
\author{中村 遼太郎}
\begin{document}

\usemintedstyle{titech}
\begin{frame}[noframenumbering, plain]
\titlepage
\end{frame}
\section{導入}
\begin{frame}[t]
  \frametitle{runcってなに}
  後にDockerから独立した低レベルなコンテナランタイムのCLI
  % 左にレイヤの図。右に箇条書
  \begin{figure}
    \centering
    \includegraphics[width=6.5cm]{images/containerd-diagram-v1.png}
    \caption{Docker, runc, OSの階層\footnote{\scriptsize{\href{https://www.docker.com/blog/containerd-vs-docker}{containerd vs. Docker: Understanding Their Relationship and How They Work Together}より}}}
    \label{fig:runc}
  \end{figure}
\end{frame}
\begin{frame}[fragile=singleslide]
  \frametitle{イメージはいらない}
  コンテナのルート\texttt{/}があればいい
  \begin{center}
    \inputminted[fontsize=\footnotesize]{sh}{code/run.sh}
    runcでコンテナの\texttt{sh}を開始\supercite{runc}  
  \end{center}
\end{frame}
\begin{frame}[t]
  \frametitle{runcの設計}
  runcの実装はUNIXプログラミング
  \begin{center}
    Infrastructure Plumbing Manifesto\footnote{\scriptsize{Dockerのブログ \href{https://www.docker.com/blog/runc/}{Introducing runC: a lightweight universal container runtime
}より引用}}
    \end{center}
  \begin{quote}
    \begin{itemize}
    \item {\small When you need to create new plumbing, make it easy to re-use and contribute improvements back.}
    \item {\small Follow the unix principles: several simple components are better than a single, complicated one.}
    \item {\small Define standard interfaces which can be used to combine many simple components into a more sophisticated system.}
  \end{itemize}
  \end{quote}
\end{frame}
\begin{frame}[t]
  \frametitle{runc runの手続き}
  runが実行したrunc initがコンテナになる
  \begin{figure}
    \centering
    \includegraphics[width=6.5cm]{images/images.drawio.pdf}
    \caption{runとinitコマンドの処理}
    \label{fig:fuga}
  \end{figure}
\end{frame}
\section{コンテナの構造体}
\begin{frame}[t]
  \frametitle{libcontainer.Container}
  フィールドは非公開
  \begin{columns}
    \begin{column}{0.5\textwidth}
      \begin{center}
        \inputminted{go}{code/container.go}
        libcontainer.Container\supercite{libcontainer}
      \end{center}
    \end{column}
    \begin{column}{0.5\textwidth}  %%<--- here
      \begin{itemize}[leftmargin=0.2cm,label=$\circ$]
        \item \texttt{stateDir}は\texttt{runc}の\texttt{--root}オプション\texttt{<dir>}とすると\texttt{<dir>/<id>}
        \item \texttt{stateDir}には\texttt{run init}を呼びだす \texttt{/proc/self/exe}の複製や名前つきパイプをおく
        \item \texttt{/proc/self/exe}は実行中のプログラムのパスのシンボリックリンク
      \end{itemize}
    \end{column}
\end{columns}
  % 設定、プロセス、リソース制御のフィールドがある
  % \begin{center}
  %   \inputminted[fontsize=\footnotesize]{go}{code/container.txt}
  %   \texttt{Container}のフィールド
  % \end{center}
  % initプロセスか
  % stateDirにfifoをおく
  % _LIBCONTAINER_FIFOFD=でつたえる
  % fifoは、fifoではなくfd/以下のファイルを親プロセスでは読みにいってスタートしたか調べる。
  % fdの説明
\end{frame}
\begin{frame}[t]
  プロセスの非機能を隔離、監視\supercite{rdt}する
  \frametitle{libcontainer.Container}
  \begin{center}
    \inputminted{go}{code/container2.go}
    libcontainer.Container\supercite{libcontainer}
  \end{center}
\end{frame}
\begin{frame}[t]
  コンテナで実行したいプロセスの情報がある
  \frametitle{libcontainer.Container}
  \begin{center}
    \inputminted{go}{code/container3.go}
    libcontainer.Container\supercite{libcontainer}
  \end{center}
\end{frame}
\begin{frame}[t]
  \frametitle{libcontainer.Container}
  コンテナのチェックポイント、リストアのフィールドがある\supercite{criu}
  \begin{center}
    \inputminted{go}{code/container4.go}
    libcontainer.Container\supercite{libcontainer}
  \end{center}
\end{frame}
\section{アタッチしたコンテナとホスト間の標準IO}
\begin{frame}[t]
  \frametitle{疑似端末の準備}
  \texttt{runc init}は\texttt{runc run}に疑似端末のマスタを送る
  \setlist[enumerate]{label*=\arabic*.,ref=\arabic*}
  
  \begin{enumerate}[leftmargin=1.2cm]
  \item \texttt{run}は\texttt{socketpair}でファイル記述子のペアをつくる。ペアは双方向通信できる
  \item \texttt{run}はCmd.ExtraFilesで記述子を1つ\texttt{init}にわたす 
  \item \texttt{init}は\texttt{/dev/ptmx}をひらき、疑似端末のマスタ/スレーブのファイル記述子をつくる。入力は他方への書き込みになる
  \item \texttt{init}はsendmsgでマスタを\texttt{run}に送る
  \item \texttt{run}はもらったマスタの入出力を標準IOにコピーする
  \end{enumerate}
\end{frame}
\begin{frame}[t]
  \frametitle{\texttt{init}がマスタを\texttt{run}に送る}
  \begin{center}
    \inputminted{go}{code/tty_init.go}
    \href{https://github.com/opencontainers/runc/blob/7cb363254b69e10320360b63fb73e0ffb5da7bf2/libcontainer/init_linux.go\#L371}{\texttt{setupConsole}}関数の抜粋
  \end{center}
\end{frame}
\begin{frame}[t]
  \frametitle{a}
  \begin{center}
    \inputminted{go}{code/tty_run.go}
    \href{https://github.com/opencontainers/runc/blob/7cb363254b69e10320360b63fb73e0ffb5da7bf2/libcontainer/init_linux.go\#L371}{\texttt{setupConsole}}関数の抜粋
  \end{center}
\end{frame}
% runcの寄贈
% https://www.docker.com/blog/runc/
% https://www.docker.com/blog/containerd-vs-docker/
% https://docs.docker.com/engine/daemon/alternative-runtimes/
\begin{frame}[t]
  \frametitle{名前つきパイプ}
  % container_linuxのincludeExec
\end{frame}

\begin{frame}[allowframebreaks]
  \frametitle{参考資料}
  \printbibliography
  % 引用していないbibファイルの要素も記載する  
  \nocite{*} 
\end{frame}

\end{document}
